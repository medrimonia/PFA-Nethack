\documentclass[10pt,a4paper]{report}
\usepackage[utf8]{inputenc}
\usepackage[french]{babel}
\usepackage{graphicx}

\begin{document}

\chapter{Utilisation du code fourni}
\section{Génération des exécutables}
TODO David/Benoit
\section{Lancer une partie avec un bot}
TODO
\section{Paramétrer l'exécution de nethack}
TODO
\section{Lancer plusieurs parties grâce à un script}
Le script \emph{game\_runner.sh} permet à l'utilisateur de lancer facilement
un grand nombre de parties qui ne diffèrent que par la graine aléatoire
utilisée.
\\
Il est possible de lancer plusieurs instances de ce script en simultané, et ce
même si elles enregistrent leurs résultats dans la même base de donnée. Chaque
script créé un dossier dans /tmp/ avant d'y copier le code de nethack ainsi que
le bot, ceci permet de continuer à travailler sur les bots ou sur le code de
nethack sans risquer d'interférer avec les parties déjà lancées. Si ce
comportement n'est pas souhaité, il est relativement simple de commenter la
partie du code qui y est associé.
\\
Plusieurs options peuvent être utilisées pour ce script, il est possible
d'obtenir plus de détails facilement en utilisant la commande suivante :
\emph{game\_runner.sh -h}.
\section{Remplir une base de donnée}
Le script \emph{data\_builder.sh} permet de générer facilement une base de
données contenant les résultats d'un grand nombre de parties. Ce script n'est
pas paramétré, mais il est aisément modifiable à la main, son but est
principalement de fournir une base pour permettre à un utilisateur novice
d'ajouter ou de supprimer des bots où des nombres de mouvements autorisés à
l'ensemble des parties à lancer.
\\
Bien que lui même ne lance pas plusieurs processus, ce script peut
parfaitement être exécuté $N$ fois, $N$ étant le nombre de processeur
disponible. Cependant l'affichage n'a pas été prévu pour gérer plusieurs
processus à l'aide d'un seul terminal, ainsi, si l'on souhaite observer
l'évolution des exécutions, il est conseillé des les exécuter dans $N$
terminaux différents.
\section{Revoir une partie}
TODO Benoit
\section{Générer des graphes}
Afin de jauger les performances des différents bots, mais aussi afin de
déceler des bugs, il peut être pratique de générer des graphes illustrant les
données qui ont été générées par les parties jouées. Ces graphes peuvent
non seulement permettre d'évaluer la répartition des résultats d'un bot,
mais aussi de comparer les performances moyennes des bots entre eux. Il existe
un dernier type de graphe qui permet de jauger la répartition des portes
secrètes et des couloirs secrets en fonction de la position dans la carte.

\subsection{Graphes analysant un bot}
Le script \emph{impulse\_graph.sh} permet de créer un grand nombre de graphes
indiquant la répartition des performances du bot, ces graphes sont générés à
l'aide de gnuplot et se présentent sous cette forme.

\begin{figure}[htb]
  \centering
  \includegraphics[width=120mm]{impulse_graph.eps}
\end{figure}

Lors de son exécution, ce script créé un dossier présent par bot trouvé dans
la base de donnée passée en paramètre et y ajoute tous les graphes le
concernant. En conséquence, il est hautement recommandé que le dossier ne
contienne que la base de donnée lorsque celui-ci est exécuté.

\subsection{Graphes comparant des bots}
Afin de comparer les performances des bots dans différents domaines, il est
possible d'utiliser le script \emph{move\_graph.sh}. Celui-ci génère des
graphes pour les principales performances générées par les bots, les résultats
se présentent sous cette forme.

\begin{figure}[htb]
  \centering
  \includegraphics[width=120mm]{move_graph.eps}
\end{figure}

\subsection{Graphes indiquant la répartition de différents objets}

\chapter{Les bots}
\section{Le starter package java}
TODO Ludovic
\section{Le bot diffusion}
TODO Ludovic
\section{Le starter package python}
TODO Benoît
\section{Le bot spécialisé}
TODO David

\chapter{Description technique}
\section{Les patchs du noyau de nethack}
\subsection{Les patchs nécessaires}
TODO hook etc
\subsection{Les patchs facultatifs}
TODO mode de jeu etc (cf doc de spec
\section{Stockage des résultats}

\chapter{Apporter ses propres modifications au code existant}
\section{Modifier du code source contenu dans les hooks}
\section{Modifier les données stockées}
\section{Ajouter de nouveau patchs}

\chapter{Dysfonctionnements connus}
\section{Sémaphore bloquée de façon permanente}
TODO Ludovic

\end{document}

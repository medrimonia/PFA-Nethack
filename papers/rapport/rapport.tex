\documentclass{article}
\usepackage[frenchb]{babel}
\usepackage[utf8]{inputenc}
\usepackage[T1]{fontenc}

\title{Des bots pour NetHack}

\begin{document}
\maketitle
\tableofcontents

\pagebreak

\section{Introduction}

NetHack est un rogue-like (jeu inspiré du jeu vidéo \emph{Rogue} - 1980)
sorti en 1987. Le joueur incarne un aventurier chargé de récupérer une
amulette dans un donjon peuplé de monstres.


\section{Environnement de création de bots}

\subsection{Prototypage}

NetHack propose, dès sa première version, une interface ASCII qui représente un donjon sous la forme suivante:

\begin{verbatim}
                                     Weapons
                                     a - a blessed +1 mace (weapon in hand)
                                     Armor
   -------        ------------       b - a +0 robe (being worn)
   .......         ........>.|       c - a +0 small shield (being worn)
   |.....|        |..........|       Comestibles
   |......#       |@.........|       e - a clove of garlic
   |<....|#       |..........+       f - a sprig of wolfsbane
   |.....|#       -.---- -----       Spellbooks
   -------############               g - a spellbook of create monster
         #############               h - a spellbook of detect food
         # #        ##               Potions
                                     d - 4 potions of holy water
                      .              Tools
                      ..             i - an oil lamp
                       ..            (end) 
                       ...
                       ...
                       ----

JohnDoe the Aspirant          St:11 Dx:14 Co:13 In:11 Wi:15 Ch:11  Chaotic
Dlvl:1  $:0  HP:14(14) Pw:8(8) AC:7  Exp:1
\end{verbatim}

Le \verb!@! désigne le joueur, les caractères \verb!|! et \verb!-! sont des murs, les \verb!#! des couloirs, un \verb!+! symbolyse une porte, etc.


\subsubsection*{Séquences d'échapement ANSI}

Cette interface fonctionne à l'aide de caractères d'échapement normalisés \footnote{http://www.inwap.com/pdp10/ansicode.txt} permettant de contrôler la position du curseur dans un terminal et d'afficher des caractères à l'endroit souhaité. Par exemple, une séquence '\verb![46;50H|...+#!' affichera la chaîne de caractères '\verb!|...+#!' en commançant à la ligne 46, colone 50. Il est alors facile de reconstituer une carte envoyée par le jeu: le '\verb!|!' est aux coordonnées (46, 50), le '\verb!+!' est situé à (46, 54), etc.
	
	Le projet TAEB (Tactical Amulet Extraction Bot) \footnote{https://github.com/sartak/TAEB} utilise ce principe pour reconstituer la carte à l'aide d'un module sachant interpréter ces séquences de caractères \footnote{https://metacpan.org/module/Term::VT102}.

\subsection{Interface finale}

\begin{itemize}
\item Choix pour l'interface (sockets, etc).
\item Protocole.
\end{itemize}

\subsection{Développement de bots.}


\section{Analyses}

Outils d'analyse, graphes, sous-section pour chaque "mode"


\end{document}

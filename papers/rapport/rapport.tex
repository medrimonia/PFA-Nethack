\documentclass{article}
\usepackage[frenchb]{babel}
\usepackage[utf8]{inputenc}
\usepackage[T1]{fontenc}

\title{Des bots pour NetHack}

\begin{document}
\maketitle
\tableofcontents

\pagebreak

\section{Introduction}

NetHack est un rogue-like (jeu inspiré du jeu vidéo \emph{Rogue} - 1980)
sorti en 1987. Le joueur incarne un aventurier chargé de récupérer une
amulette dans un donjon peuplé de monstres.


\section{Environnement de création de bots}

\subsection{Prototypage}

NetHack propose, dès sa première version, une interface ASCII qui représente un donjon sous la forme suivante:

\begin{verbatim}
                                     Weapons
                                     a - a blessed +1 mace (weapon in hand)
                                     Armor
   -------        ------------       b - a +0 robe (being worn)
   .......         ........>.|       c - a +0 small shield (being worn)
   |.....|        |..........|       Comestibles
   |......#       |@.........|       e - a clove of garlic
   |<....|#       |..........+       f - a sprig of wolfsbane
   |.....|#       -.---- -----       Spellbooks
   -------############               g - a spellbook of create monster
         #############               h - a spellbook of detect food
         # #        ##               Potions
                                     d - 4 potions of holy water
                      .              Tools
                      ..             i - an oil lamp
                       ..            (end) 
                       ...
                       ...
                       ----

JohnDoe the Aspirant          St:11 Dx:14 Co:13 In:11 Wi:15 Ch:11  Chaotic
Dlvl:1  $:0  HP:14(14) Pw:8(8) AC:7  Exp:1
\end{verbatim}

Le \verb!@! désigne le joueur, les caractères \verb!|! et \verb!-! sont des murs, les \verb!#! des couloirs, un \verb!+! symbolyse une porte, etc.


\subsubsection*{Séquences d'échapement ANSI}

Cette interface fonctionne à l'aide de caractères d'échapement normalisés \footnote{http://www.inwap.com/pdp10/ansicode.txt} permettant de contrôler la position du curseur dans un terminal et d'afficher des caractères à l'endroit souhaité. Par exemple, une séquence '\verb![46;50H|...+#!' affichera la chaîne de caractères '\verb!|...+#!' en commançant à la ligne 46, colone 50. Il est alors facile de reconstituer une carte envoyée par le jeu: le '\verb!|!' est aux coordonnées (46, 50), le '\verb!+!' est situé à (46, 54), etc.
	
	Le projet TAEB (Tactical Amulet Extraction Bot) \footnote{https://github.com/sartak/TAEB} utilise ce principe pour reconstituer la carte à l'aide d'un module sachant interpréter ces séquences de caractères \footnote{https://metacpan.org/module/Term::VT102}. Une première approche fut donc de copier ce procédé.

\subsubsection*{Pseudo-terminal}

Pour empêcher certaines formes de tricheries, NetHack procède à quelques vérifications pour vérifier qu'il est bien lancé depuis un terminal, ce qui empêchent les redirections de son entrée/sortie. Les quelques lignes responsables peuvent être désactivées sans conséquence notable sur le reste du jeu mais cela nécessite de modifier le code original. Pour rediriger à la fois l'entrée et la sortie du jeu sans modifier le jeu, il est nécessaire de 'tromper' NetHack à l'aide d'un pseudo terminal.

Un pseudo terminal (ptry) est une paire de pseudo périphériques dont l'un est appelé 'maître' et l'autre 'esclave'. Le maître est utilisé comme un terminal standard sur lequel on peut écrire ou lire du texte. L'esclave communique avec l'application et sert simplement de rapporteur entre la partie 'maître' et NetHack. Ainsi, le jeu est satisfait puisqu'il est lancé depuis un terminal et il peut maintenant être contrôlé depuis la partie 'master' du pty sur laquelle on souhaite brancher un bot.

\subsection{Interface finale}

Décomposition, archi, hooks, choix sur les glyphes (ambiguitées), systèmes de
patchs (simplicité, intégration).

\begin{itemize}
\item Choix pour l'interface (sockets, etc).
\item Protocole.
\end{itemize}

\subsection{Développement de bots.}

BDD, outils d'analyse, généricité (protocole, socket, différents langages).

Algos de bots.


\section{Analyses}

graphes, sous-section pour chaque "mode". Reprendre étude pièce rectangulaire.


\section{Travail en équipe, organisation}

Utilisation de git.
Comparer avec le travail en plus grande équipe.

\end{document}
